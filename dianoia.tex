\documentclass{report}
\usepackage{titlesec}
\usepackage{titling}
\usepackage{graphicx}
\usepackage{tikz}
\usepackage{pgfplots}
\usepackage{caption}
\usepackage{multirow}


\title{\textbf{Dianoia - The Pravega Debate Competition}}
\author{Rahul Chavan and Aditya Kamath Ammembal}
\date{20th November 2023}


\usepackage[left=1in, right=1in, top=1in, bottom=1in]{geometry}
\renewcommand{\maketitle}{
 \begin{center}
    \includegraphics[width=2cm]{IISc_Master_Seal_Black.jpg}
    \vspace{0.5cm}

    \Large
    \textbf{\thetitle}
    
    \vspace{0.5cm}
    
    \Large
    \theauthor
    
    \vspace{0.2cm}
    
    \large
    \thedate

    \vspace{0.5cm}

    \hrule  
    
  \end{center}
}

\pgfplotsset{compat=1.18}
\begin{document}
\maketitle
\begin{center}
    \Large
    \textbf{Anchoring Script}
\end{center} 

\begin{center}
    \large
    \textbf{Introduction}
\end{center}

Good morning, respected judges, participants, and our enthusiastic audience! 
A warm and patriotic welcome to "Dianoia," the first edition of debate competition of pravega, the UG fest at IISc, 
unfolding on this remarkable day of 26th of January, our Republic Day.

As we gather on this momentous occasion, we not only celebrate the intellectual prowess showcased in the serene campus of IISc 
but also pay homage to the democratic ideals that bind us as a nation, as a republic. The echoes of our Constitution resonate today, 
not just in the preamble but in the spirited debates that will unfold, echoing the very essence of democracy.

If you are wondering why Dianoia, it is a Greek word that means "thinking" or "thought", and aptly
encapsulates the spirit of our journey today. It is a quest for knowledge, a celebration of diverse perspectives, 
and a tribute to the intellectual fervor that characterizes all of us here.

Before we embark on this journey, let us extend our heartfelt gratitude to our judges for taking time out of their busy schedules
to be with us today.

We are honored to have Dr. Anupama Sharma Ma'am, an Assistant Professor at the Institute of Communication and Media Studies, St. Joseph's University,
specialising in media laws and ethics. 

We are also delighted to have with us, Dr. Bitasta Das Ma'am, an author, educator, TEDx speaker, Member-State Planning Commission,for 
the state of Chattisgarh and Senior Editor at the Office of Communications, IISc.


And we also have Ullas Aparanji, a PhD student at the Department of Computer Science and Automation, IISc, who has represented IISc in competitions such as IICM,
in literary events such as story writing. He has also published a compilation of his poems titled Broken Promises.

We are also grateful to our audience for joining us today, we hope that you will enjoy the competition.

This competition was divided into two rounds. The preliminary round, which was online presented us with a plethora of ideas and opinions.
It was very impressive that all the participants were able to present their views in a concise and coherent manner.

The top 8 participants were selected for the final round, which is being held today. 
Let me breifly introduce the format of the competition.
There will be 5 rounds in total, each round will be a test of the participants' oratory skills,
logical reasoning, and ability to think on their feet. The first round will be a simple introduction round. The second round will be a parliamentary debate, 
the third round will be a group discussion, the fourth round will be an extempore, and the final round will be a moderated debate.

The rules of each round will be explained before the round begins. The participants will be judged on the basis of their performance in each round.

Eliminations will take place after the third and fourth rounds, i.e, after the group discussion 4 participants will be eliminated and after the extempore 
1 participant will be eliminated.

The final round will be an interrupted debate between the top 3 participants.

Let us now begin our journey,

\begin{center}
    \large
    \textbf{Round 1}
\end{center}
\subsubsection*{Introduction}
The first round is called "Meet the Finalists". Each finalist will step into the spotlight, 
introducing themselves to us. Beyond names and academic backgrounds, we request the participants to offer glimpses into 
the passions and dreams that drive their pursuits.

The rules for the first round are rather simple, each participant will be given 3 minutes to introduce themselves.

I now invite the event coordinator Chethana to give some more details.

We will invite the participants now, in no particular order, we request them to adhere to the time limit.


Let us now commence - 'Dianoia'


\subsubsection*{concluding}
That was fascinating! We are delighted to have such a diverse set of participants, and we wish them the very best for the upcoming rounds.

\begin{center}
    \large
    \textbf{Round 2}
\end{center}

\subsubsection*{Introduction}
The second round is a parliamentary debate, where we hope to be enlightened by the participants' arguments, and the clashes of ideas that will ensue.
We will have four teams, each team will have two participants. The participants will be given a motion, and we 
expect them to present their arguments in a concise and coherent manner. The participants will be judged on the basis of their
arguments, their ability to think on their feet, and their ability to respond to questions from the judges and the audience.

The rules for the second round are as follows, we request the judges to also make a note of these rules.

The participants will be divided into 4 pairs by chits.
A coin will be tossed to decide which pair will be for the motion and which pair will be against the motion.
Then the motion will be displayed on the screen. The participants will be given 5 minutes to prepare their arguments.
Each participant will be given 1.5 minutes to present their opening statements. Then there will be a 1.5 minute rebuttal round.
After that, the judges and the audience will be given a minute to ask questions to the participants.
Then the participants will be given 1 minute each to present their concluding remarks.

Please note that the participants can use mobile phones to prepare their arguments, but they cannot use them during the debate.
The participants can also use pen and paper to take notes during the debate.

Again, we hope that the participants will adhere to the time limit. We will now invite the participants for the second round.


\subsubsection*{Concluding}
Eloquent! 

\begin{center}
    \large
    \textbf{Round 3}
\end{center}

\subsubsection*{Introduction}
The third round is a group discussion. Participants, in this 10-minute segment, you will navigate a sea of diverse opinions on a given topic. 
The ability to articulate and defend your viewpoint while respecting others' perspectives will be key. 

The rules for the third round are as follows:
The participants will be divided into two groups of 4 each by chits.
As soon as the topic is displayed on the screen for a group, the participants will be given 3 minutes to think about the topic.
Then the participants will be given 10 minutes to discuss the topic. 

This will be followed by a 5 minute question and answer session, where the judges and the audience can ask questions to the participants.

Participants, are to again make a note of the fact that they can use mobile phones to prepare their arguments, but they cannot use them during the discussion.
The participants can use pen and paper to take notes during the discussion.

We remind both the judges and the participants that this is an elimination round, 4 participants will be eliminated after this round.

We will now invite the participants for the third round.

\subsubsection*{Concluding}
Those were really interesting discussions!

We will now request the judges to select 4 participants for elimination and announce their results.


Congratulations! We now have our top 4 finalists.
For the others, it is unfortunate that we must bid an adeiu, in the competition, but we hope that you will continue to participate in such events in the future.


\begin{center}
    \large
    \textbf{Round 4} 
\end{center}

\subsubsection*{Introduction}
We move on to our next round - the extempore. This will be a test of the participant's spontaneity and unscripted brilliance.

The rules are simple, the participants will be given a topic, they will have 1 minute for contemplation, and then 
they will have 3 minutes to speak about the topic.

We will all brace ourselves for a showcase of intellectual agility.

Let me remind you again that this is the penultimate round, 1 participant will be eliminated after this round.

We will now invite the participants for the fourth round.

\subsubsection*{Concluding}
That was a riveting round! 

.
We will now take a short break for lunch, and resume the competition after 1 hour.
Hope to see you all here at 2:00 pm.


\begin{center}
    \large
    \textbf{Round 5}
\end{center}



We will now request the judges to select 1 participant for elimination and announce their results.
Congratulations! We now have our top 3 finalists

We welcome you back to the final round of Dianoia, 
where we will witness an interrupted debate between the top 3 finalists.

Congratulations again to the finalists, we are delighted to have you here.

The rules for the final round are as follows:

We will do this in two sets, with largely similar formats.

For the first set:

The participants will be given 3 minutes to think about the topic.
Then the participants will be given 7 minutes to speak about the topic. The participants will be allowed to interrupt each other.
This will be followed by a 5 minute question and answer session, where the participants can ask questions to each other.

For the second set:

Again the participants will be given 3 minutes to think about the topic.
Then the participants will be given 7 minutes to speak about the topic. The participants will be allowed to interrupt each other.
This will be followed by a 5 minute question and answer session, where the judges and audience can question the participants.

This will mark the end of our competiton, we request the participants to give their best. 

We will display the topic now and invite the finalists on the stage.


\begin{center}
    \large
    \textbf{Conclusion}
\end{center}

Finally as we wrap up "Dianoia," it is woth mentioning that this wasn't just a competition, but a vibrant exchange of ideas. 
Each participant brought their unique perspective, enriching our discussions and debates.
A special shout-out to our finalists: your hard work and strong beliefs have shone through. You've shown what it means to stand up for 
your ideas and speak your mind.
The discussions we've had here showed the importance of appreciating different views and respectful debate.

I would like to thank our judges for their time and valuable feedback.
I would also like to thank our sponsor airbus, the event coordinators chethana and aditi, volunteers, culutural coordinators of pravega, and everyone else
who helped us out with the logistics etc.
I would like to thank our judges for their time and valuable feedback.
And finally to all finalists and their well-wishers, thank you for being a part of this event.

We hope that you enjoyed the event, and we hope to see you again next year.
Thank you and have a great day!

\end{document}
