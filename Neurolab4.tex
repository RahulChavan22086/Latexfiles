\documentclass{report}
\usepackage{titlesec}
\usepackage{titling}
\usepackage{graphicx}
\usepackage{tikz}
\usepackage{pgfplots}
\usepackage{caption}
\usepackage{multirow}


\title{\textbf{UBL201-L Introductory Biology III - Neurobiology}}
\author{Rahul Chavan - 22086}
\date{20th November 2023}


\usepackage[left=1in, right=1in, top=0.5in, bottom=0.5in]{geometry}
\renewcommand{\maketitle}{
 \begin{center}
    \includegraphics[width=2cm]{IISc_Master_Seal_Black.jpg}
    \vspace{0.5cm}

    \Large
    \textbf{\thetitle}
    
    \vspace{0.5cm}
    
    \Large
    \theauthor
    
    \vspace{0.2cm}
    
    \large
    \thedate

    \vspace{0.5cm}

    \hrule  
    
  \end{center}
}

\pgfplotsset{compat=1.18}
\begin{document}
\maketitle
\begin{center}
    \Large
    \textbf{Psychophysics and cognition}
\end{center} 

%For your assignment, you need to describe the experiment, tell about the phenomenon you are testing and interpret the results you got.
%Put all of these with the plots you have generated in a PDF and submit in the Lab 4 teams assignment channel. Mind your group!
%Name your report as "[GroupID]_[last five digits of your S.R.No.]_UB201L4.pdf".  For Example: "GroupA_14889_UB201L4.pdf
%Color Pop-out Task
%Orientation Pop-out Task
%Color-Orientation Conjunction Task
%Color-Color Conjunction Task
%Working Memory Task
%Priming Task

\subsection*{The experiment}
This experiment is an introduction to psychophysics and cognition.

Psychophysics quantitatively investigates the relationship between 
physical stimuli and the sensations and perceptions they produce. 
It has been described as the study of the 
relation between stimulus and sensation or, more completely, as 
the analysis of perceptual processes by studying the effect on a 
subject's experience or behavior by systematically varying the 
properties of a stimulus along one or more physical dimensions.

Cognition is the mental action or process of acquiring knowledge 
and understanding through thought, experience, and the senses. 
It encompasses many aspects of intellectual functions and processes 
such as attention, the formation of knowledge, memory and working 
memory, judgment and evaluation, reasoning and computation, problem 
solving and decision making, comprehension and production of language. 
Cognitive processes use existing knowledge and generate new knowledge.

In this experiment we study cognitive processes like visual 
attention, working memory and priming. The experiment consists 
of 6 tasks, each of which is designed to test a different aspect 
of cognition.  The tasks are as follows:




\begin{enumerate}
    \item Color Pop-out Task
    \item Orientation Pop-out Task
    \item Color-Orientation Conjunction Task
    \item Color-Color Conjunction Task
    \item Working Memory Task
    \item Priming Task
\end{enumerate}

The first four tasks are visual search tasks.

Visual search is a type of perceptual task requiring 
attention that typically involves an active scan of the 
visual environment for a particular object or feature 
(the target) among other objects or features (the distractors). 
It has a limited capacity for information processing and the 
ability to filter out unwanted information. The first two tasks 
are feature search tasks; the next two tasks are conjunction search 
tasks.

The last two tasks are memory tasks that test the working memory 
and priming. Priming is a phenomenon whereby exposure to one stimulus 
influences a response to a subsequent stimulus, without conscious 
guidance or intention. Working memory is a cognitive system with a 
limited capacity that is responsible for temporarily holding information 
available for processing.

We will infer the cognitive processes from the reaction time and 
accuracy of the subject in performing the tasks.




%things to keep in mind while interpreting the results:
%1. set size affect your accuracy and/or reaction time
%2. difference in accuracy/reaction time in the case of presence or absence of targets.
%3. why pop-ot takes less time than conjuction.


\subsubsection{1. Color Pop-out Task}

The color pop-out effect is a phenomenon in which a single feature, 
such as color, is sufficient to guide attention to a target item, 
regardless of the number of distractors or the similarity of the 
distractors to the target. In our task we observed that one of the 
objects pops out because of its dissimilarity to the other items, 
which have the same color.




\begin{figure}[htbp]  
  \centering 
  \includegraphics[width=1.0\textwidth]{Rahul Chavan_1_Colour-pop_10-Nov-2023.jpg} 
  \caption{Plots for color pop-out task.}
  \label{fig: colour pop-out} 
\end{figure}

This is bottom-up processing where the visual system can detect 
these properties in parallel across the visual field. Here, the 
number of distractors does not matter in both the cases where the 
target is present and where the target is absent. This is concluded 
by the fact that the reaction time and accuracy (probability correct) 
is almost constant for different set sizes. A dip in the probability 
correct for large set size is observed in the target present condition. 
This could be due to the fatigue (impatience) of the subject and is not 
of significant interest.


\subsubsection{2. Orientation Pop-out Task}

The orientation pop-out effect is a phenomenon in which a single 
feature, such as orientation, is sufficient to guide attention to a 
target item, regardless of the number of distractors or the 
similarity of the distractors to the target.


\begin{figure}[htbp]  
  \centering 
  \includegraphics[width=1.0\textwidth]{orient-pop.jpg} 
  \caption{Plots for orientation pop-out task}
  \label{fig: orientation pop-out} 
\end{figure}

In our task, when the color, length, and thickness of both 
the target and distractors are same but only the orientation of 
the distractors is different, the target pops out. Even with 
distributed attention, a single target object should be detected 
in the same amount of time, no matter how many distractors are present

In fact, the reaction time in target present condition decreased 
in the sets with more distractors. This could be due the 
fact that the distractors are more similar to each other than to 
the target, making it easier to detect the target and also because 
the subject is more attentive in the target present condition and 
has strategized to recognize it faster, which should not be hard as 
the target continuously pops out. 

Here again, the number of distractors 
does not matter in both the cases where the target is present and where 
the target is absent, which is concluded by the fact that the reaction 
time and accuracy (probability correct) is almost constant for different 
set sizes (except for some errors in target absent condition leading 
to a dip in probability correct and can be accounted for by subject fatigue).



\subsubsection{3. Color-Orientation Conjunction Task}

The color-orientation conjunction effect is a phenomenon in which a 
single feature, such as color or orientation, is not sufficient to 
guide attention to a target item, but both features are required to 
guide attention to a target item. Here, the target is dissimilar 
from distractors in both color and orientation.


\begin{figure}[htbp]  
  \centering 
  \includegraphics[width=1.0\textwidth]{conj-pop.jpg} 
  \caption{Plots for color-orientation conjunction task}
  \label{fig: color-orientation conjunction} 
\end{figure}

We observe that on increasing the set size, the reaction time increases. 
This is serial search where we search for individual objects and try to 
match with the target until we exhaust all possibilities to conclude 
that the target is not present in the display. 

We also observe that errors occur at large set sizes in both target 
present and target absent conditions. Also, the accuracy takes a dip 
in both the conditions at large set sizes. Clearly, the number of 
distractors does matter here. This is concluded by the fact that 
the reaction time is not constant and is gradually increasing with 
an increase in set sizes.






\subsubsection{4. Color-Color Conjunction Task}

The color-color conjunction effect is a phenomenon in which a 
single feature, such as color, is not sufficient to guide attention 
to a target item, but a combination of two colors is required to 
guide attention to a target item. Here we are looking for a 
particular color-color combination where red top - green bottom 
rectangle is the target whilst the opposite are the distractors.

\begin{figure}[htbp]  
  \centering 
  \includegraphics[width=1.0\textwidth]{Conj2-pop.jpg} 
  \caption{Plots for color-color conjunction task}
  \label{fig: color-color conjunction} 
\end{figure}

We observe that this task in general has a higher reaction time. 
Also, the reaction time increases rapidly with an increase in set 
size in both target present and target absent conditions, which 
concludes that the number of distractors do matter here. 
The accuracy, however, is not affected much by the set size 
in both conditions. This is another example for serial search. 
Searching for an opposite orientation of the two colors in a pool 
of distractors with same colors is a difficult task, accounting for 
the higher reaction time.



\vspace{1cm}

We can in general also conclude that the reaction time is higher in 
conjunction search tasks than in feature search (pop-out) tasks. This is because 
in feature search tasks, the target pops out and is easy to detect and 
the visual field can be scanned in parallel, whereas in conjunction 
search tasks, the target does not pop out and the visual field has to be 
scanned serially to detect the target, causing a higher reaction time, 
and making the task difficult. The accuracy should also ideally 
decrease in conjunction search tasks as the distractors are more 
similar to the target than in feature search tasks, but this is not 
observed in our experiment and can be explained by the subject's 
strategy (like compensating with a higher reaction time) and 
attentiveness.


\subsubsection{5. Working Memory Task}

Working memory is a cognitive system with a limited capacity 
that is responsible for temporarily holding information available 
for processing. 

\begin{figure}[htbp]  
  \centering 
  \includegraphics[width=1.0\textwidth]{Memory-search.jpg} 
  \caption{Plots for working memory task}
  \label{fig: working memory} 
\end{figure}

In our task, we were asked to remember the letters presented on 
the screen and then search for their presence or absence on the 
next screen. We observe that the reaction time increases with an 
increase in set size in both target present and target absent conditions. 
The number of errors and the reaction time for those errors also increases with an 
increase in set size. We can also see that accuracy takes a dip with the increase 
in set size in both target present and target absent conditions. 
This is because the working memory has a limited 
capacity and the more the number of items to be remembered, the more difficult 
it is to remember them.



\subsubsection{6. Priming Task}

Priming is a phenomenon whereby exposure to one stimulus influences 
a response to a subsequent stimulus, without conscious guidance or 
intention. For example, the word NURSE is recognized more quickly 
following the word DOCTOR than following the word BREAD. Priming 
can be perceptual, semantic, or conceptual. It can also be positive, 
negative, or affective in nature.


\begin{figure}[htbp]  
  \centering 
  \includegraphics[width=1.0\textwidth]{Rahul Chavan_1_Priming_10-Nov-2023.jpg} 
  \caption{Plots for priming task}
  \label{fig: priming} 
\end{figure}

In our task, we were supposed to identify the presence of red 
squares among green squares and green squares among red squares.

If there is a sequence of red squares among green squares 
followed by a green square among red squares or vice versa, 
such trials are called switch trials.  Here, the targets may or 
may not appear in the same position.

What is interesting to note is that the reaction time for switch trials 
is higher than that for non-switch trials. This is because the subject 
is primed to look for a particular color and the reaction time is less 
when the target appears at the same position. The reaction time is more 
when the target appears at a different position as the subject has to 
search for the target at all the positions. The accuracy is also less 
for switch trials than for non-switch trials. These can also vary with 
the subject's strategy and attentiveness.


\end{document}