\documentclass{article}
\usepackage{float} 
\usepackage{bookmark}
\usepackage{titlesec}
\usepackage{titling}
\usepackage{graphicx}
\usepackage{tikz}
\usepackage{pgfplots}
\usepackage{caption}
\usepackage{multirow}
\usepackage{fancyhdr} 
\usepackage{graphicx}
\usepackage{hyperref}
\usepackage{titlesec}
\usepackage{natbib}
\usepackage{amsmath}
\usepackage{amsthm}
\usepackage{amssymb}
\usepackage{parskip}
\usepackage[utf8]{inputenc}
\usepackage{multicol}
\usepackage{csvsimple}
\usepackage{longtable}


\title{\textbf{Purification of Green Fluorescent Protein (GFP)}}
\author{Rahul Chavan }
\date{\today}
\markboth{Protein}{righthead}


\usepackage[left=1in, right=1in, top=1in, bottom=1in]{geometry}
\renewcommand{\maketitle}{
 \begin{center}
    
        \includegraphics[width=2cm]{IISc_Master_Seal_Black.jpg}
        \vspace{0.5cm}

        \Large
        \textbf{\thetitle}
        
        \vspace{0.5cm}
        
        \Large
        \theauthor
        
        \vspace{0.2cm}
        
        \large
        \thedate

        \vspace{0.5cm}

        \hrule  
        
    \end{center}
}

\pgfplotsset{compat=1.18}
\begin{document}
\maketitle



\begin{abstract}
    \textbf{Rahul Chavan\footnote{Department of Biochemistry, Indian Institute of Science, Bangalore, India. Email:} and Dr. Rahul Chavan\footnote{Department of Biochemistry, Indian Institute of Science, Bangalore, India. Email:}}
Green Fluorescent Protein (GFP) is a protein that exhibits bright green fluorescence when exposed to light in the blue to 
ultraviolet range. GFP is a 238 amino acid protein, which is composed of 11 beta sheets that form a barrel surrounding a 
single alpha helix. The chromophore is formed from the tripeptide Ser65-Tyr66-Gly67. The chromophore is responsible for 
the green fluorescence of GFP. The GFP gene was cloned into a plasmid and transformed into \textit{E. coli}. 
The transformed \textit{E. coli} was grown in LB medium and induced with IPTG. The cells were lysed and the GFP 
was purified using Ni-NTA affinity chromatography. The purified GFP was analysed using SDS-PAGE and UV-Vis spectroscopy. 
The purified GFP showed a single band at 27 kDa on SDS-PAGE and a peak at 395 nm on UV-Vis spectroscopy. 
The purified GFP was also analysed using fluorescence spectroscopy. The purified GFP showed a peak at 510 nm on 
fluorescence spectroscopy. The fluorescence emission spectrum of the purified GFP was also recorded. The fluorescence 
emission spectrum of the purified GFP showed a peak at 510 nm. 
\end{abstract}


\begin{multicols}{2}

\subsection*{Introduction}
The purification of proteins is an essential for the following reasons:
\begin{itemize}
    \item To study the structure and function of the protein.
    \item To study the interactions of the protein with other molecules.
    \item To study the interactions of the protein with other proteins.
    \item To study the interactions of the protein with other cells.
    \item To study the interactions of the protein with other organisms.
    \item To study the X-ray crystallography of the protein.
    \item To study the NMR spectroscopy of the protein.
    \item To study the mass spectrometry of the protein.
    \item To study the proteomics of the protein.    
\end{itemize}
The purification of proteins is a multistep process. The purification of proteins involves the following steps:
\begin{itemize}
    \item Cell lysis
    \item Centrifugation
    \item Precipitation
    \item Chromatography
    \item Dialysis
    \item Concentration
    \item Storage
    \item Analysis
    \item Characterization
\end{itemize}

\subsection*{Aims and Objectives}
The aims and objectives of this experiment are as follows:


\subsection*{Materials and Methods}
\subsubsection*{Materials}
The materials used for this experiment are as follows:


\subsubsection*{Methods}
The methods used for this experiment are as follows:



\subsection*{Results}
\subsubsection*{SDS-PAGE}
The SDS-PAGE gel of the purified GFP is shown in Figure . The purified GFP showed a single band at 27 kDa on SDS-PAGE.


\subsection*{Discussion}
The GFP gene was cloned into a plasmid and transformed into \textit{E. coli}. 
The transformed \textit{E. coli} was grown in LB medium and induced with IPTG. 
The cells were lysed and the GFP was purified using Ni-NTA affinity chromatography. 
The purified GFP was analysed using SDS-PAGE and UV-Vis spectroscopy. 
The purified GFP showed a single band at 27 kDa on SDS-PAGE and a peak at 395 nm on UV-Vis spectroscopy. 
The purified GFP was also analysed using fluorescence spectroscopy. 
The purified GFP showed a peak at 510 nm on fluorescence spectroscopy. 
The fluorescence emission spectrum of the purified GFP was also recorded. 
The fluorescence emission spectrum of the purified GFP showed a peak at 510 nm.

\subsection*{Conclusion}


\subsection*{Future Directions}

\subsection*{Acknowledgements}
I would like to thank my lab partner, Dr. Rahul Chavan, for his help and support during this experiment.




\subsection*{References}



\end{multicols}






\end{document}
