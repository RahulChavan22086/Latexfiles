\documentclass[a4paper, 12pt]{article}

\usepackage[utf8]{inputenc}
\usepackage{amsmath}
\usepackage{amsfonts}
\usepackage[a4paper]{geometry}
\usepackage{enumitem}
\usepackage{color}
\usepackage{hyperref}
\usepackage{booktabs}
\usepackage{titlesec}

\hypersetup{
	colorlinks=true, 
	linktoc=all,     
	linkcolor=blue,
	linktocpage
}

\titleformat{\section}[display]{\Large\bfseries\itshape}{}{0ex}
{\rule{\textwidth}{0.3pt}\vspace{0ex}
	\centering}[\vspace{-1ex}\rule{\textwidth}{0.3pt}]

\titleformat{\subsection}[display]{\large\itshape}{}{0ex}{}[]

\newcommand{\unnumberedsection}[1]{
	\newpage
	\phantomsection
	\addcontentsline{toc}{section}{#1}
	\section*{#1}}
\renewcommand\thesubsection{\Roman{subsection}}

\parindent0mm
\parskip1.5ex plus0.5ex minus0.5ex



\title{\rule{\textwidth}{1.5pt}\vspace{0ex}
	Laboratory Records
\vspace{-1ex}\rule{\textwidth}{1.5pt}}
\author{Rahul Chavan}
\date{Fall 2023}

\begin{document}
	
	\begin{titlepage}
		\maketitle
		\thispagestyle{empty}
		\newpage
		\tableofcontents
		\thispagestyle{empty}
	\end{titlepage}
	
	\unnumberedsection{2nd october 2023}
    \begin{center} \Large
        \textbf{Gel extraction of PCR products}
    \end{center} 

	\subsubsection*{Principle}

	
	\unnumberedsection{4th october 2023}
    \begin{center} \Large
        \textbf{Transformation of yeast cells}
    \end{center} 

	\subsubsection*{Principle}

   The competent cells are prepared fresh everytime before a transformation.
   These fresh competent cells are then mixed with the plasmid DNA and then subjected to heat shock.
   The heat shock causes the cells to take up the plasmid DNA.
   The cells are then resuspended in PBS and plated on SC-His plates.
   This is done as the deletion is done using His cassettes.
   The deletions carried out are as follows:
   \begin{itemize}
    \item 009 - c-terminus - HOF1
    \item 502 - c-terminus - HOF1
    \item 502 - coil coil domain 
    \item 502 - $SH_{3}$ domain.
   \end{itemize}


   \subsubsection*{General Protocol}
   \begin{itemize}
    \item Add 7.5 $\mu$l of plasmid DNA to 50 $\mu$l of competent cells.
    \item Incubate in the hood for 15 minutes.
    \item Add 300 $\mu$l of Li-PEG 
    \item Incubate in the hood for 15 minutes.
    \item Add 30 $\mu$l of DMSO.
    \item Heat shock at 42$^{\circ}$C in dry bath for 15 minutes.
    \item Centrifuge at 3200 rpm for 3 minutes.
    \item Resuspend the pellet in 120 $\mu$l of PBS.
   \end{itemize}

   \unnumberedsection{16th october 2023}
    \begin{center} \Large
        \textbf{Master Class}
    \end{center} 

    The heat shock treatment for a temperature sensitive mutant is done at 30$^{\circ}$C overnight with DMSO. This is the lazy method.
    \begin{itemize}
        \item there are 3 collections in the lab - yeast, bacteria, plasmid.
        \item whenever a strain is made, it is stored in the -80$^{\circ}$C freezer after adding it in 50\% glycerol 1 ml.
        \item For s. pombe add yea+ 15\% glycerol 1 ml.
        \item bacterial culture 50\% glycerol 500 $\mu$l + 1 ml of culture.
        \item whenever a strain is made it can stay on the plate for maximum of 5 days.
    \end{itemize}

    \subsection*{Yeast Transformation}
    We use S. cervisiae generally in the lab.
    there are two mother strains - S288C and W303.
    We use S288C in the lab and its commpetent cells can be used for a long time.
    However, W303 is not so good and its competent cells have to be made fresh everytime.
    As W303 is a genetically unstable strain.
    BY4741 is also widely used.
    
    whenever strains are taken out of the -80 freezer, be quick and put them back as soon as possible.
    Dont overuse -80 freezer.
    Yeast transformations require Li-PEG and Li-sorb.
    Li-PEG used in the lab is - 3350
    Li-PEG and Li-Sorb are filter sterilized.
    Sterile PBS.
    ESM356 is a derivative of S288C and is white in colour.
    YPH499 is pink in colour also a derivate of S288C , colour due to lack of adenine.

X, Y, Z = (transformation of a) plasmid, tagging, deletion
to overexpress- change the promoter, or put it in a vector which can take multiple copies.
tagging - lets say we want to tag gene A with GFP/HA/mcherry
one has to do 4 transformations- one for gfp, one for ha, one for mcherry and one for negative control.
There are around 4993 orf in yeast genome.



yeast- 1990's announced as model organism.
EUROSCARF - collection of all the strains.
PYM - where P stands for plasmid, Y stands for yeast and M stands for Michael.
Yeast has high level of homologous recombination.


    

\end{document}
