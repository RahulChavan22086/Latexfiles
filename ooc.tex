\documentclass[12pt]{article}
\usepackage{titlesec}
\usepackage{titling}
\usepackage{graphicx}
\usepackage{subcaption}
\usepackage{tikz}
\usepackage{caption}
\usepackage{csquotes}




\title{\textbf{ Science Communication Internship}}

\author{Rahul Chavan - 22086}
\date{20th January 2024}


\usepackage[left=1in, right=1in, top=0.5in, bottom=0.5in]{geometry}
\renewcommand{\maketitle}{
 \begin{center}
    \includegraphics[width=1.5cm]{IISc_Master_Seal_Black.jpg}
    \vspace{0.5cm}

    \Large
    \textbf{\thetitle}
    
    \vspace{0.3cm}
    
    \large
    \theauthor
    
    \vspace{0.2cm}

    
    
    \large
    \thedate
    \vspace{0.2cm}

    \hrule  
    
    
  
  \end{center}
}


\begin{document}
\maketitle
  \begin{center} \Large
    \textbf{Science/Engineering inspired by nature}
  \end{center} 

Science and Engineering are inspired by nature. Nature has been a source of inspiration for many scientists and engineers.
The field of biomimicry is a result of this inspiration. 
Biomimicry is the imitation of the models, systems, and elements of nature for the purpose 
of solving complex human problems. The term biomimicry and biomimetics come from the Greek words 
bios, meaning life, and mimesis, meaning to imitate. The term biomimicry was coined by Janine Benyus 
in her 1997 book Biomimicry: Innovation Inspired by Nature. The book was a result of her research for a 
film on the topic of biomimicry, done by her in 1995, with Leonardo DiCaprio as the narrator.

Be it the invention of the aeroplane or the invention of the Velcro, nature has been a source of inspiration for many scientists and engineers.
The Wright brothers were inspired by the flight of birds and the aerodynamics of the birds.
The invention of the Velcro was inspired by the burrs of the burdock plant.
Even the invention of the bullet train was inspired by the beak of the kingfisher bird.
The making of strong and light materials like carbon fibre is inspired by the structure of the bones of birds.
The invention of the solar cell was inspired by the photosynthesis process in plants.
The invention of the self-cleaning paint was inspired by the lotus leaf.

We have always learned and we keep trying to gain more. We although have also destroyed a lot of nature in the process of learning.



\end{document}

