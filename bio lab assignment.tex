\documentclass{article}
\usepackage{titlesec}
\usepackage{titling}
\usepackage{graphicx}



\title{\textbf{UBL201-L Introductory Biology III - Immunology}}
\author{Rahul Chavan - 22086}
\date{12th September 2023}


\usepackage[left=1in, right=1in, top=0.5in, bottom=0.5in]{geometry}
\renewcommand{\maketitle}{
 \begin{center}
    \includegraphics[width=2cm]{IISc_Master_Seal_Black.jpg}
    \vspace{0.5cm}

    \Large
    \textbf{\thetitle}
    
    \vspace{0.5cm}
    
    \Large
    \theauthor
    
    \vspace{0.2cm}
    
    \large
    \thedate
\vspace{0.2cm}

    \hrule  
    
    
  
  \end{center}
}


\begin{document}
\maketitle
    \begin{center} \Large
        \textbf{The various organs of the Immune System in the Mouse}
    \end{center} 
\subsection*{Question 1:} \textbf{Explain what are primary and secondary lymphoid organs and their role in the immune response.}

\textbf{Answer:} The immune system comprises of several morphologically and functionally diverse organs and tissues that contribute to the development 
of immune responses.  On the basis of function they are classified as \textbf{\textit{primary}} and \textbf{\textit{secondary}} lymphoid organs.
    \begin{itemize}
      \item \textbf{Primary lymphoid organs:} The \textit{thymus} and the \textit{bone marrow} are the primary lymphpoid organs. These are the 
      sites within which immature lymphocytes generated in hematopoiesis mature and become committed to a particular antigenic specificity.They 
      ensure immunocompetency of the lymphocytes. 
        \begin{itemize}
          \item \textbf{Bone marrow:} The bone marrow is a primary lymphoid organ responsible for the production 
          of various immune cells, including B lymphocytes (B cells) and some T lymphocytes (T cells).
           B cells mature in the bone marrow, where they develop receptors for specific antigens, which are molecules that can trigger an immune response.
          \item \textbf{Thymus:} The thymus is a bilobed organ situated above the heart. 
          It is the site of T-cell development and maturation. 
          T-cells originate in the bone marrow but migrate to the thymus to undergo a selection process that 
          ensures they can distinguish between self and non-self antigens. 
          The thymus helps shape the T cell receptor repertoire and educates T cells to be tolerant to self-antigens while responsive to foreign invaders. 
          
        \end{itemize}
      \item \textbf{Secondary lymphoid organs:} The \textit{spleen}, \textit{lymph nodes}, \textit{tonsils}, \textit{mucosal associated lymphoid tissue (MALT)}, \textit{Peyer's patches}, \textit{adenoids} and \textit{appendix}, etc.
      are the secondary lymphoid organs.  Secondary lymphoid organs are sites where immune responses are initiated and coordinated. When immune cells encounter antigens in these organs,
       they undergo activation and clonal expansion, leading to the production of effector cells 
       (e.g., antibody-producing plasma cells and cytotoxic T cells) that can target and eliminate pathogens.
        Additionally, secondary lymphoid organs facilitate the interaction between different immune cells,
         enhancing the efficiency and specificity of the immune response.
      
    \end{itemize}


\subsection*{Question 2:} \textbf{What does this experiment show and is useful for?}

\textbf{Answer:}  This experiment which was conducted on mice shows the various organs of the immune system in the mouse.
 The dissection of the mouse involved the removal of the thymus, inguinal lymph nodes, mesenteric lymph nodes and spleen. 
 Then we also performed spleen perforation to isolate spleenocytes.

    This experiment is useful to understand the location of various immunological organs in the body and also their function in the immune response.
    The video also makes us familiar with handling of mouse and the intricacies of dissection of the mouse for immunological studies.




\subsection*{Question 3:} \textbf{For what purpose spleen is used in immunological assays and why?}

\textbf{Answer:} Spleen is a secondary lymphoid organ.  It is located in the upper left quadrant of the abdomen,
  just beneath the diaphragm. It is a highly vascular organ and is responsible for the filtration of blood. 
  It is the site of immune responses to blood-borne pathogens, antibody production, immune cell activation and proliferation. The spleen is a reservoir of immune cells, including macrophages, dendritic cells, 
  B cells, and T cells which makes it an essential source for isolation of these cells.
  Since it is also the site where interactions between antigens and immune cells occur, it also serves as a source for studying immune response.
  It is the largest lymphoid organ in the body, hence easily accessible and becomes useful in immunological assays.
  




\subsection*{Question 4:} \textbf{What if lymphoid organs are removed, how would this affect immunity in mouse?}

\textbf{Answer:}  The lymphoid organs are the sites where the immune response is initiated and coordinated.
  The removal of lymphoid organs would affect the immune response in the mouse as it would not be able to mount
  an immune response against the pathogen and would be susceptible to infections.
  
  It will be Impaired in its ability to recognize antigens and activate an immune response, resulting in the following:
    \begin{itemize}
      \item  Impaired Antigen Recognition and Activation
      \item Reduced Antibody Production
      \item T Cell Deficiency
      \item Impaired Immune Memory
      \item Increased Susceptibility to Infections
      \item Autoimmune Risk
      \item Impaired Wound Healing
      \item Impaired Cancer Immunosurveillance
      \item Impaired Immune Response to Vaccines
      \item Impaired Immune Response to Allergens
    \end{itemize}
   Hence, the mouse would be immunocompromised and would succumb to the simplest of infections.




\subsection*{Question 5:} \textbf{What is a nude mouse and explain its use in immunology?}

\textbf{Answer:} A ``nude mouse'' is a laboratory mouse strain that is characterized by a genetic mutation that results 
in a lack of a functional thymus gland, which in turn leads to a severe immunodeficiency. This genetic mutation is 
caused by a recessive autosomal mutation in the Foxn1 gene. The nude mouse is named as such because it lacks a typical
 fur coat, giving it a hairless appearance.

The absence of a functional thymus in nude mice results in a lack of T lymphocytes (T cells) in their immune system. 
T cells are a critical component of the immune system and play a central role in cell-mediated immunity, including the 
recognition and elimination of virus-infected cells and cancer cells. Since nude mice have a deficiency in T cells, their 
immune responses, especially those related to cellular immunity, are severely compromised.

The use of nude mice in immunology research is significant and valuable for several reasons:
\begin{itemize}
  \item Tumor studies
  \item Xenograft studies
  \item Stem cell research
  \item Autoimmune and infectious disease research
  \item Drug development and testing.
\end{itemize}


\end{document} 

