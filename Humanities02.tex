\documentclass{report}
\usepackage{titlesec}
\usepackage{titling}
\usepackage{graphicx}
\usepackage{tikz}
\usepackage{pgfplots}
\usepackage{caption}
\usepackage{multirow}
\usepackage{amsmath}

\title{\textbf{UH201 - Ways of Doing : Mapping Science-Society Relationship}}
\author{Rahul Chavan - 22086}
\date{\textbf{Assignment 2}}


\usepackage[left=1in, right=1in, top=0.5in, bottom=0.5in]{geometry}
\renewcommand{\maketitle}{
 \begin{center}
    \includegraphics[width=2cm]{IISc_Master_Seal_Black.jpg}
    \vspace{0.5cm}

    \Large
    \textbf{\thetitle}
    
    \vspace{0.5cm}
    
    \Large
    \theauthor
    
    \vspace{0.2cm}
    
    \large
    \thedate

    \vspace{0.5cm}

    \hrule  
    
  \end{center}
}

\pgfplotsset{compat=1.18}
\begin{document}
\maketitle
\begin{center}
    \Large
    \textbf{Law and Science}
\end{center} 

\subsubsection*{1. What are the three key factors that should be considered by an
innovator or researcher or scientist when applying for a patent?
Explain with examples.}

To maimize the chances of getting a patent, an innovator or researcher or scientist should consider the following three key factors:

\subsubsection*{Novelty}
Novelty is a fundamental requirement for obtaining a patent. It ensures that the invention is genuinely new and hasn't 
been disclosed or publicly known before the filing date. This extends to any form of public use, sale, or publication, 
anywhere in the world. Conducting comprehensive prior art searches is crucial to identify existing patents, scientific 
literature, or any public disclosures related to the invention. The goal is to establish that the proposed innovation 
brings something genuinely new to the table.

For example, imagine one has developed a groundbreaking holographic display for smartphones. To meet the novelty 
requirement, one would need to search for any existing patents or publications related to holographic displays in 
smartphones. If only the search reveals no similar technologies, the invention is more likely to satisfy the novelty criterion and will be 
accepted for patenting.

\subsubsection*{Utility}
Utility is about demonstrating that the invention has a practical use and can be applied in some kind of industry (say). 
This ensures that the invention is not merely theoretical but has tangible real-world applications. The utility 
requirement is met when an invention provides a specific, credible, and substantial benefit. It addresses the 
question of why the invention matters and how it can be practically employed.

For example, suppose one has invented a new material that can be used to create stronger and lighter aircraft 
components. The utility of the invention lies in its industrial applicability, contributing to the aerospace 
industry's need for materials that enhance aircraft performance. Clearly articulating the practical uses and 
benefits of the invention enhances the utility aspect of the patent application.

\subsubsection*{Non-obviousness (Inventive step)}
Non-obviousness goes beyond mere novelty; it requires that the invention involves an inventive step not 
obvious to someone skilled in the relevant field. This criterion ensures that the innovation demonstrates 
a level of creativity or ingenuity that goes beyond what experts in the field would consider straightforward 
or expected. It's about more than just being new; it's about being innovative in a way that is not 
immediately apparent to those with expertise in the subject matter.

For instance, consider a software developer creating a new algorithm for data compression. To meet the 
non-obviousness requirement, the algorithm should involve a unique and unexpected approach that goes beyond 
standard compression techniques. Demonstrating an inventive step strengthens the patent application by 
showcasing the innovation's creative aspects.


In addition to these factors, a successful patent application should include a detailed description of the invention. 
This involves outlining the problem the invention solves, how it works, and the specific benefits it offers over 
existing technologies. The more comprehensive and well-documented the application, the better the chances of successfully obtaining a patent.


\subsubsection*{2. How should any technology or technological invention be
assessed from the point of view of a stakeholder viz. (i) Economy
or Industry (ii) User (iii) State or Govt. and (iv) Society?}


\subsubsection*{\textbf{(i)Economy or Industry}}
From the standpoint of economy or industry, it becomes important to consider factors such as 
its economic impact, market potential, innovation and research, societal implications, regulatory 
compliance, environmental sustainability, collaboration opportunities, and return on investment. 
Assessing job creation, productivity gains, global competitiveness, and ethical considerations 
provides insights into the technology's overall contribution to economic growth and societal 
well-being. Additionally, a focus on compliance with regulations, environmental sustainability, 
and fostering industry collaboration enhances the likelihood of successful adoption and positive outcomes for stakeholders.

\subsubsection*{\textbf{(ii)User}}
The user's standpoint involves prioritizing user-friendliness, functionality, 
accessibility, and overall value. Ensuring an intuitive interface, addressing specific user needs, and providing 
ample support contribute to a positive user experience. Factors like data privacy, affordability, and adaptability 
also play crucial roles in user acceptance. Regular feedback mechanisms and metrics help refine the technology 
based on user insights, fostering an ongoing cycle of improvement. Ultimately, a successful assessment considers 
not only the technological features but also the seamless integration of the technology into the user's daily 
activities, resulting in enhanced satisfaction and adoption.


\subsubsection*{\textbf{(iii)State or Govt.}}
From the government's perspective it involves evaluating its impact on public welfare, 
regulatory compliance, and economic growth. Governments prioritize technologies that contribute to national 
security, create jobs, and integrate seamlessly with existing infrastructure. Additionally, considerations 
for resource efficiency, digital inclusion, and data privacy are crucial. Governments play a role in fostering 
innovation through research and development initiatives, encouraging public-private partnerships, 
and ensuring emergency preparedness. A successful assessment from the government's standpoint aligns 
the technology with broader socio-economic goals, promoting its responsible use for the benefit of the 
citizens and the country as a whole.

\subsubsection*{\textbf{(iv)Society}}
The society's standpoint involves examining its impact on inclusivity, ethics, privacy, and well-being. 
The technology should provide equitable access, respect cultural diversity, and safeguard individual privacy rights. 
Attention to physical and mental health effects, community engagement, and environmental sustainability are crucial. 
Empowering communities through skill development and ensuring accessibility for vulnerable groups contribute to positive 
societal outcomes. Transparency, accountability, and a focus on long-term societal impact build public trust. 
A successful assessment from society's perspective prioritizes the positive contributions of the technology to 
individuals and communities while mitigating potential risks or adverse effects.


\subsubsection*{3. Write an essay on your view of AI-based science and technology
regulation and policy for India in the present and coming future.}
\begin{center}
  Advancing AI-based Science and Technology Regulation and Policy in India: Current Status and Future Prospects
\end{center}

The 21st century has witnessed an unprecedented surge in the development and deployment 
of artificial intelligence (AI) and related technologies. As India seeks to position itself 
as a global leader in technological innovation, the regulation and policy framework surrounding 
AI-based science and technology have become imperative for the country's sustainable growth and 
competitive edge. 

India, with its strong IT infrastructure, skilled workforce, and burgeoning startup ecosystem, 
has made significant strides in the AI domain. The government's vision to create a conducive 
environment for the growth of AI and related technologies is evident in initiatives like 
the National Strategy for Artificial Intelligence, launched in 2018, which aims to harness 
AI for social and economic well-being. The policy framework in place includes guidelines for data protection 
and privacy, technology standards, and ethical AI deployment. Additionally, the establishment of institutions 
like the Centre of Excellence for Artificial Intelligence and Robotics and the National Mission on Interdisciplinary 
Cyber-Physical Systems underscores the government's commitment to fostering AI-driven innovation.
However, despite these efforts, challenges persist in the regulatory and policy landscape. 
India currently lacks a comprehensive legislative framework dedicated solely to AI and emerging 
technologies. The absence of specific regulations poses challenges in addressing issues such as 
data governance, algorithmic transparency, and ethical considerations, leaving the door open 
for potential misuse and ethical dilemmas. Furthermore, the rapid pace of technological advancements 
often outpaces the development of regulatory policies, leading to a gap between innovation and governance.

Policies and regulations can be formed along the following lines:

\begin{itemize}
  \item With the increasing volume of data generated and processed by AI systems, 
        the protection of individual privacy and data governance has become a pressing concern. 
        India needs robust data protection laws that balance the promotion of innovation with the safeguarding of individual rights and privacy.
  \item AI algorithms have the potential to perpetuate biases and discrimination, 
        making it crucial to integrate ethical considerations into the regulatory framework. Promoting fairness, 
        accountability, and transparency in AI systems should be a priority for policymakers to build public 
        trust and confidence in AI technologies.
  \item Bridging the gap between the demand for AI skills and the existing 
        talent pool is essential for the sustainable growth of the sector. Policy initiatives focusing on 
        AI education, training, and skill development will be instrumental in fostering a skilled 
        workforce capable of driving AI innovation and research.
  \item Strengthening intellectual property rights protection 
        for AI-based innovations is vital to incentivize research and development in the field. A robust IPR 
        framework can encourage investment in AI, leading to a competitive advantage for Indian companies in the global market.
  \item Engaging in international collaborations and adhering to global 
        AI standards can help India stay at the forefront of technological advancements. Aligning national 
        AI policies with global best practices will facilitate interoperability and foster a conducive 
        environment for international partnerships and investments.

\end{itemize}



Certainly, the development of AI policies and regulations is a global phenomenon, 
with many countries taking proactive measures to address the challenges and opportunities 
presented by AI technologies. Several nations have implemented diverse strategies to foster 
AI innovation while ensuring ethical and responsible AI deployment. For example:

European Union (EU) - General Data Protection Regulation (GDPR): The GDPR, implemented in 2018, 
is one of the most comprehensive data protection regulations globally. While not specific to AI, 
the GDPR has significant implications for AI development, emphasizing the protection of personal 
data and the rights of individuals. It places strict regulations on the collection, storage, and 
use of personal data, ensuring that AI systems built on EU data comply with stringent privacy standards. Or the
United States - National Artificial Intelligence Research and Development Strategic Plan: The United States has 
taken significant strides in advancing AI through various initiatives, including the National AI Research and 
Development Strategic Plan. The plan focuses on prioritizing investments in AI research and development, 
promoting the integration of AI across various sectors, and addressing ethical and societal implications. 
Additionally, the U.S. has established the National Artificial Intelligence Initiative Office to coordinate 
AI-related activities across federal agencies.

India can draw inspiration from these initiatives and develop a comprehensive AI policy framework.
India stands at a critical juncture in its journey towards becoming a global AI powerhouse. 
The formulation and implementation of effective AI-based science and technology regulation 
and policy will play a pivotal role in shaping the country's technological landscape and 
ensuring sustainable growth. By addressing the existing challenges and capitalizing on the 
opportunities presented by AI, India can position itself as a frontrunner in the global AI 
revolution, contributing to economic prosperity and societal well-being. A holistic approach, 
encompassing ethical considerations, data governance, skill development, and international 
collaboration, will be instrumental in fostering a robust AI ecosystem for the present and the future.









\end{document}