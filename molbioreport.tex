\documentclass{report}
\usepackage{titlesec}
\usepackage{titling}
\usepackage{graphicx}
\usepackage{tikz}
\usepackage{pgfplots}
\usepackage{caption}
\usepackage{multirow}
\usepackage{amsmath}

\title{\textbf{UBL201-T Introductory Biology III - Molecular Biology}}
\author{Rahul Chavan - 22086}
\date{11th November 2023}


\usepackage[left=1in, right=1in, top=0.5in, bottom=0.5in]{geometry}
\renewcommand{\maketitle}{
 \begin{center}
    \includegraphics[width=2cm]{IISc_Master_Seal_Black.jpg}
    \vspace{0.5cm}

    \Large
    \textbf{\thetitle}
    
    \vspace{0.5cm}
    
    \Large
    \theauthor
    
    \vspace{0.2cm}
    
    \large
    \thedate

    \vspace{0.5cm}

    \hrule  
    
  \end{center}
}

\pgfplotsset{compat=1.18}
\begin{document}
\maketitle
\begin{center}
    \Large
    \textbf{Telomeres and Telomerase}
\end{center} 

\subsection*{Introduction}
%Define telomeres and telomerase in simple terms.
%Explain their significance in maintaining genome stability and cell function.
%Highlight the importance of resolving the end replication problem.

The ends of linear chromosomes are capped by specialized structures called telomeres. 
Telomeres are composed of repetitive DNA sequences and associated proteins that protect 
the ends of chromosomes from degradation and fusion. Telomeres also play a key role in 
regulating the cellular lifespan and preventing the onset of cancer. Telomerase is an 
enzyme that maintains telomere length by adding repetitive nucleotide sequences to the 
ends of chromosomes. Telomerase is highly expressed in embryonic stem cells and cancer cells, 
but is absent in most somatic cells. Telomerase activation is a key step in the development of cancer. 
Telomerase is also a potential target for anti-aging therapies and regenerative medicine. 
In this report, we will discuss the structure and function of telomeres and telomerase, 
and their implications in aging and cancer. We will also explore the potential applications 
of telomerase in regenerative medicine and anti-aging therapies.






\subsection*{Understanding Telomeres and Telomerase}
%Discuss the structure and function of telomeres.
%Explain the role of telomerase in maintaining telomere length and integrity.
%Describe the connection between telomeres, telomerase, and aging.

the telomere is a region of repetitive nucleotide sequences at each end of a chromosome,
which protects the end of the chromosome from deterioration or from fusion with neighboring chromosomes.
Its name is derived from the Greek nouns telos "end" and meros "part".
For vertebrates, the sequence of nucleotides in telomeres is TTAGGG, with the complementary DNA strand being AATCCC, with a single-stranded TTAGGG overhang.
This sequence of TTAGGG is repeated approximately 2,500 times in humans.
In humans, average telomere length declines from about 11 kilobases at birth to less than 4 kilobases in old age,
telomers are also present in prokaryotes, but are less well-studied.







\subsection*{The End Replication Problem: causes and implications.}
%Explain the end replication problem in DNA replication.
%Discuss how the conventional DNA replication machinery cannot fully replicate the ends of linear chromosomes.

the end replication problem is a problem that is faced during the replication of linear chromosomes.
The problem is that DNA replication occurs in a 5' to 3' direction, and the DNA helicase that unwinds the DNA double helix at the replication fork moves along the template strand in the 3' to 5' direction.
The antiparallel nature of DNA means that the DNA polymerase responsible for synthesizing new strands, can only add nucleotides in the 5' to 3' direction.
As a result, the DNA polymerase synthesizes a new strand of DNA in a 5' to 3' direction, but it can only do so continuously along one of the two strands.
Along the other strand, the DNA polymerase must work in the direction away from the replication fork, and it must therefore synthesize DNA in short fragments, known as Okazaki fragments, which are subsequently joined together by DNA ligase.
This mechanism of DNA replication is known as semi-conservative replication.
The problem is that the last Okazaki fragment cannot be fully replicated, because there is no upstream primer for the DNA polymerase to bind to and synthesize the final Okazaki fragment.
This results in the loss of a small portion of DNA at the end of each chromosome with each round of DNA replication.
This is known as the end replication problem.
The end replication problem is a major cause of aging and cancer, as it leads to the gradual shortening of telomeres with each round of DNA replication.
Telomeres are repetitive nucleotide sequences at the ends of chromosomes that protect the ends of chromosomes from degradation and fusion.
Telomeres also play a key role in regulating the cellular lifespan and preventing the onset of cancer.
Telomerase is an enzyme that maintains telomere length by adding repetitive nucleotide sequences to the ends of chromosomes.


\subsection*{Resolving the End Replication Problem: telomerase a key solution}
%Describe the discovery of telomerase and its role in solving the end replication problem.
%Explain how telomerase maintains telomere length by adding repetitive nucleotide sequences to the ends of chromosomes.
%Discuss the implications of telomerase activation and its connection to cancer and aging.

telomerase is a ribonucleoprotein that adds a species-dependent telomere repeat sequence to the 3' end of telomeres.
A telomere is a region of repetitive sequences at each end of eukaryotic chromosomes in most eukaryotes.
Telomeres protect the end of the chromosome from DNA damage or from fusion with neighbouring chromosomes.
Its name is derived from the Greek nouns telos "end" and meros "part".
For vertebrates, the sequence of nucleotides in telomeres is TTAGGG, with the complementary DNA strand being AATCCC, with a single-stranded TTAGGG overhang.





\subsection*{References}
%List all the references used in the report in the appropriate format.
\begin{enumerate}
    \item Shay, J.W., Wright, W.E. Telomeres and telomerase: three decades of progress. Nat Rev Genet 20, 299-309 (2019). https://doi.org/10.1038/s41576-019-0099-1
    \item Monaghan P, Ozanne SE.
    2018 Somatic growth and telomere dynamics
    in vertebrates: relationships, mechanisms and
    consequences. Phil. Trans. R. Soc. B 373:
    20160446.
    http://dx.doi.org/10.1098/rstb.2016.0446



\end{enumerate}




\end{document}
