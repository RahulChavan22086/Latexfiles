\documentclass{report}
\usepackage{titlesec}
\usepackage{titling}
\usepackage{graphicx}
\usepackage{tikz}
\usepackage{pgfplots}
\usepackage{caption}
\usepackage{multirow}
\usepackage{amsmath}

\title{\textbf{UBL201-L Introductory Biology III - Molecular Biology}}
\author{Rahul Chavan - 22086}
\date{26th October 2023}


\usepackage[left=1in, right=1in, top=0.5in, bottom=0.5in]{geometry}
\renewcommand{\maketitle}{
 \begin{center}
    \includegraphics[width=2cm]{IISc_Master_Seal_Black.jpg}
    \vspace{0.5cm}

    \Large
    \textbf{\thetitle}
    
    \vspace{0.5cm}
    
    \Large
    \theauthor
    
    \vspace{0.2cm}
    
    \large
    \thedate

    \vspace{0.5cm}

    \hrule  
    
  \end{center}
}

\pgfplotsset{compat=1.18}
\begin{document}
\maketitle
\begin{center}
    \Large
    \textbf{Plasmid DNA extraction from E.coli using Alkaline lysis method.}
\end{center} 

%Lab report: 
%   Aim of the experiment
%	Principle
%	Results and Discussion: (i) Photograph of the gel electrophoresis with all bands clearly labelled
%         (ii) Concentration of DNA and 260/280 ratio
%        (iii) Discuss the results obtained and speculate on the potential source of error.
%   Answer the following questions:

\section*{Aim of the experiment:}
\begin{itemize}
  \item To extract plasmid DNA from E.coli cells using Alkaline lysis method.
\end{itemize}

\section*{Principle:}
Plasmid DNA is the extra chromosomal DNA present in bacteria. It is circular in shape and is used as a vector for cloning genes. 
The plasmid DNA extraction protocol is based on the alkaline lysis method. 
We use three solutions for the extraction process:
\begin{itemize}
  \item Solution I: Resuspension Buffer - This contains Tris, EDTA, glucose, and RNase A.
  Tris is a buffering agent and EDTA is a chelating agent that binds to Mg\textsuperscript{2+} ions and inhibits the activity of DNases.
  Glucose helps in maintaining the osmotic pressure and RNase A degrades RNA.
  \item Solution II: NaOH and SDS - NaOH denatures the DNA and SDS denatures the proteins.
  \item Solution III: Potassium acetate - It neutralizes the pH and precipitates the proteins.
\end{itemize}

we have used the QIAprep® Spin Miniprep Kit, which is a spin column-based kit for the isolation of plasmid DNA from bacterial cells.
The kit uses silica membrane technology to purify plasmid DNA.
The plasmid DNA binds to the silica membrane in the presence of high salt concentration and is eluted in a low salt buffer.

\section*{Results and Discussion:}
We quantified the DNA using a Nanodrop spectrophotometer and the concentration of the DNA was found to be 351.1 ng/$\mu$l.
The 260/280 ratio was found to be 1.86, which is close to the ideal value of 1.8, and the 260/230 ratio was found to be 2.16, 
which is close to the ideal value of 2.0. This indicates that the DNA is pure and is free from contaminants(proteins and organic compunds or chaotropic salts respectively).
The DNA was then run on an agarose gel and the gel was visualized under UV light.
All observations are reported under the images section.

\section*{Images:}
\begin{figure}[!ht]
    \centering
    \includegraphics[width=10cm]{nanodrop.jpg}
    \caption*{Figure 1: Nanodrop spectrophotometer readings.} 
\end{figure}

\begin{figure}[!ht]
    \centering
    \includegraphics[width=10cm]{gelresults.jpg}
    \caption*{Figure 2: Agarose gel electrophoresis of the plasmid DNA.}
\end{figure}
The 2nd well is the ladder and the 4th well is my plasmid DNA.
The different bands in the plasmid DNA are the different topological forms of the plasmid DNA.
The supercoiled form is the most compact form of the plasmid DNA and migrates the fastest and is hence the bottom most band.
The open circular form is less compact and migrates slower than the supercoiled form and is hence the middle band.
The linear form is the least compact and migrates the slowest and is hence the top most band.

We can see that the plasmid DNA is pure and is free from contaminants as there are no prominent smears in the gel.
Hence, we can conclude that the plasmid DNA extraction was successful.

\section*{Sources of error:}
The plasmid DNA may not completely dissolved in the elution buffer. This could be due to the fact that the elution buffer was not preheated to 70\textsuperscript{o}C before adding it to the column.
This could lead to a lower concentration of the plasmid DNA.
There could also be some errors in the spectrophotometer readings due to which the concentration of the plasmid DNA could be lower than the actual value.
There may also be some error in handling as my plasmid DNA was the only sample that showed a slight smear in the gel and also has the least concentration.

\section*{Answer the following questions:}

\subsubsection*{Question 1:}
\textbf{What happens to genomic DNA during plasmid isolation that is also present in bacterial cells?}

\subsubsection*{Answer:}
The genomic DNA is denatured by the NaOH and SDS in solution II, however, unlike plasmid DNA, it does not reanneal in the presence of potassium acetate in solution III.
It is scrambled and cannot be separated from the proteins and SDS and hence is precipitated along with the proteins.

\subsubsection*{Question 2:}
\textbf{How will vortexing after adding solution II affect the quality of your plasmid DNA?}

\subsubsection*{Answer:}
Solution II contains NaOH and SDS which denature the DNA and proteins respectively.
Vortexing after adding solution II will lead to shearing of genomic DNA which will contaminate the plasmid DNA.
This will lead to a lower concentration of the plasmid DNA.

\subsubsection*{Question 3:}
\textbf{Explain the principle behind salt and isopropanol precipitation of DNA, some protocol says ice-cold isopropanol increase the yield of DNA. Comment on how temperature influences precipitation.}

\subsubsection*{Answer:}
The principle behind salt and isopropanol precipitation of DNA is that
the addition solvents such as isopropanol to an aqueous solution
reduces the dielectric constant of the medium. This reduces the solubility of the DNA molecules in
the solution, and the DNA molecules come together and form a solid precipitate which can be collected by centrifugation. Temperature influences precipitation in this process because
lower temperatures favor the formation of DNA aggregates. At lower temperatures, the solubility
of DNA in the aqueous solution is highly reduced which is
why protocols often specify the use of ice-cold isopropanol.



 

\end{document}
