\documentclass{article}
\usepackage{titlesec}
\usepackage{titling}
\usepackage{graphicx}



\title{\textbf{UBL201-L Introductory Biology III - Immunology}}
\author{Rahul Chavan - 22086}
\date{14th September 2023}


\usepackage[left=1in, right=1in, top=0.5in, bottom=0.5in]{geometry}
\renewcommand{\maketitle}{
 \begin{center}
    \includegraphics[width=2cm]{IISc_Master_Seal_Black.jpg}
    \vspace{0.5cm}

    \Large
    \textbf{\thetitle}
    
    \vspace{0.5cm}
    
    \Large
    \theauthor
    
    \vspace{0.2cm}
    
    \large
    \thedate
\vspace{0.2cm}

    \hrule  
    
    
  
  \end{center}
}


\begin{document}
\maketitle
    \begin{center} \Large
        \textbf{Applications of flow cytometry}
    \end{center} 

%Immunophenotyping​
%Blood cancers​
%DNA ploidy, cell cycle analysis​
%CD3 / CD8 counting​
%Detection of surface/intracellular markers or proteins ​
%Cell apoptosis​
%Cell proliferation​

\textit{{\fontsize{12}{20}\selectfont{Flow cytometry is used to count and analyse the size, shape and properties of individual cells within a heterogeneous population of cells.
Flow cytometry data is extremely quantitative and can be analysed in depth by specific flow cytometry software programs.
Flow cytometry is useful in a wide range of applications, which include:}}}

\section*{Cell cycle analysis}
Flow cytometry is used to study the cell cycle, which is crucial for understanding cell division, growth, and replication.
By staining DNA with a fluorescent dye, we can determine the DNA content in each phase of the cell cycle (G1, S, G2, and M).
As we know that dysregulation of the cell cycle is a hallmark of cancer, studying cell cycle abnormalities helps identify 
potential targets for cancer therapies and it can also be used to screen and develop drugs
that target specific phases of the cell cycle. 

Thus flow cytometry information of actively dividing cells is valuable for cancer research, drug development, and cell biology studies.

\section*{Immunophenotyping}
Immunophenotyping is the analysis of heterogeneous populations of cells for the purpose of 
identifying the presence and proportions of the various populations of interest.
Flow cytometry is used to identify and quantify the different types of cells in the sample
by labelling the cells with antibodies that are specific to certain cell types.
The antibodies are conjugated to fluorophores, which emit light at specific wavelengths when excited by a laser.
The fluorophores are detected by the flow cytometer and the data is analysed to determine the number of cells that are positive for each antibody.
Immunophenotyping using flow cytometry is an extremely efficient method in identifying and sorting cells within complex populations, for example the
analysis of immune cells in a blood sample. 

Immunophenotyping using flow cytomety is regularly used both in basic research and clinical laboratories.


\section*{RNA detection and analysis}
Flow cytometry can be employed to quantify RNA content in individual cells.
 This is often done using fluorescent dyes, such as thiazole orange, 
 which bind to RNA molecules. By measuring the fluorescence intensity of stained cells,
we can estimate RNA levels within a cell population. Flow cytometry can also be used to assess mRNA expression, study non-coding RNAs, 
such as microRNAs (miRNAs) or long non-coding RNAs (lncRNAs) at the single-cell level. Specific fluorescent probes or molecular beacons
can be designed to target these RNA molecules, enabling their detection and quantification in single cells.

Flow cytometry, combined with cell sorting capabilities (FACS), allows the isolation of cells based on their RNA content. 
Cells with high or low levels of specific RNA molecules can be separated for further analysis or culture,
providing invaluable information about rare cell populations or the study of RNA-related diseases.


\section*{Cell apoptosis}
Apoptosis is a form of programmed cell death that is essential for normal development and tissue homeostasis.
It is also a key mechanism of cell death in many diseases, including cancer.
Flow cytometry is a powerful technique for studying apoptosis and cell death by enabling the detection
and quantification of various apoptotic markers, such as phosphatidylserine exposure, mitochondrial membrane potential changes,
and caspase activation. It allows us to distinguish between early and late apoptotic cells, assess DNA fragmentation,
and perform multiparametric analysis, making it a versatile tool for understanding the mechanisms and dynamics of cell death processes.

\section*{Detection of surface/intracellular markers or proteins}
Flow cytometry is essential for the detection of surface and intracellular markers or 
proteins in biological samples. It helps us to analyze and quantify specific molecules
at the single-cell level. For surface markers, fluorescently labeled antibodies can be used to target
proteins present on the cell's surface, providing information about cell identity and phenotype, like in the case of immunophenotyping. In the
case of intracellular markers or proteins, tools like permeabilization and fixation are
employed to access the cell interior, enabling the detection of intracellular molecules like cytokines,
transcription factors, or signaling proteins. 


Flow cytometry's ability to analyze multiple markers
simultaneously in complex cell populations makes it invaluable in immunology, oncology, and various other
fields for understanding cell function, differentiation, and disease states.




\end{document}