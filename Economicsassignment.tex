\documentclass{article}
\usepackage{titlesec}
\usepackage{titling}
\usepackage{graphicx}
\usepackage{subcaption}
\usepackage{tikz}
\usepackage{caption}
\usepackage{csquotes}
\usepackage{amsmath}




\title{\textbf{UH201 - WAYS OF DOING: MAPPING SCIENCE-SOCIETY RELATIONSHIP}}

\author{Rahul Chavan - 22086}
\date{3rd December 2023}


\usepackage[left=1in, right=1in, top=0.5in, bottom=0.5in]{geometry}
\renewcommand{\maketitle}{
 \begin{center}
    \includegraphics[width=1.5cm]{IISc_Master_Seal_Black.jpg}
    \vspace{0.5cm}

    \Large
    \textbf{\thetitle}
    
    \vspace{0.3cm}
    
    \large
    \theauthor
    
    \vspace{0.2cm}

    
    
    \large
    \thedate
    \vspace{0.2cm}

    \hrule  
    
    
  
  \end{center}
}


\begin{document}
\maketitle
  \begin{center} \Large
    \textbf{Economics}
  \end{center} 

\subsection*{Question 1:}
\textbf{In each of the following cases, discuss the impacts on GDP;}
\begin{itemize}
    \item \textbf{Company A builds Rs. 200 worth of chocolate bars, but consumers only buy Rs. 100 of them.}
    \item \textbf{A spends Rs. 1800 on new paper-weights to use in her business. The paper-weights were produced in china.}
\end{itemize}
GDP is the total monetary value of all final goods and services produced within a country over a specific time period.
We know that,
\begin{equation}
    \text{GDP} = \text{Consumption} + \text{Investment} + \text{Government Purchases} + \text{Net Exports}
\end{equation}

\begin{itemize}
    \item In the first case, the company A builds Rs. 200 worth of chocolate bars, but consumers only buy Rs. 100 of them.
          Thus, the consumption is Rs. 100, and the unslod chocolate bars which become a part of the inventory is the investment i.e., Rs. 100 and the net exports is Rs. 0.
          Thus, the GDP is Rs. 200.
          Hence, there is an increase in the GDP by Rs. 200.
    \item In the second case, A spends Rs. 1800 on new paper-weights to use in her business which were produced in china.
          Thus, the consumption is Rs. 1800, the investment is Rs. 0 and the net exports is Rs. -1800 (as it is an import)
          Thus, the GDP is Rs. 0.
          Hence, there is no change in the GDP.
\end{itemize}


\subsection*{Question 2:}
\textbf{Briefly explain the methods of comparing the GDPs of two currencies with different currencies.}
There are two widely used methods of comparing the GDPs of two currencies with different currencies.
\begin{itemize}
    \item \textbf{Market Exchange Rate:} The market exchange rate is the rate at which the currencies of some two countries 
          can be exchanged.
          The GDP of a country can be converted to the GDP of another country by multiplying the GDP of the first country 
          with the market exchange rate of the currencies of the two countries.
    \item \textbf{Purchasing Power Parity:} The purchasing power parity is the rate at which the currencies of some two countries 
          can be exchanged such that the exchange has the same purchasing power in both the countries.
\end{itemize}
Although international organisations use any of the two approaches, I believe purchasing power parity is a better measure of 
the GDP of a country as compared to the market exchange rate due to its stability over time and it being a better measure of the
overall standard of living of the people of the country.




\subsection*{Question 3:}
\textbf{What can offset the impact of population growth on economic growth in the Solow growth model and how? 
Restrict your answer within the context of the Solow model.}

The Solow growth model is one of the neoclassical model of economic growth.
The Solow growth model assumes that there is no involvement of the government in the economy (no taxes).
It's central idea is that the economic growth is dependent on the capital stock of the economy and the savings rate of the economy,
which is summarised by the following equation:
\begin{equation}
    \Delta K = sf(K) - \delta K 
\end{equation}
where, $\Delta K$ is the change in the capital stock, $s$ is the savings rate, $f(K)$ is the production function, $\delta$ is the depreciation rate and $K$ is the capital stock.

Population growth can affect the economic growth in the Solow growth model in two ways:
\begin{itemize}
    \item \textbf{Increase in the labour force:} An increase in the population will increase the labour force of the economy.
          This will increase the production function and thus, the economic growth.
    \item \textbf{Decrease in the savings rate:} An increase in the population will decrease the savings rate of the economy.
          This will decrease the change in the capital stock and thus, the economic growth.
\end{itemize}

Therefore, the impact of population growth on economic growth in the Solow growth model can be offset by increasing 
the savings rate of the economy.
Because, an increase in the savings rate will increase the change in the capital stock and thus, lead to 
economic growth.



\subsection*{Question 4:}
\textbf{During the 1950s and 1960s, Germany and Japan had much fatser rates of economic growth than did the United States.
What might account for these differences in growth rates.}

The time period of 1950s and 1960s happens to be soon after the second world war where both Germany and Japan were
the axis powers and were defeated by the allied powers. 
After the war, both the countries were in a state of economic depression and had to rebuild their economies.
This accounts for the high growth rates in the economies of both the countries.

However, the United States was not disastrously affected by the war and did not have to rebuild its economy.
Thus, the growth rate of the United States was not as high as that of Germany and Japan.

If we think in terms of Walt Rostow's 5 stages of grwoth, we can draw parallels between the stages of growth of the economies of the countries.
Germany and Japan were in the take-off stage of growth (or maybe drive to maturity), while the United States was in the maturity stage and inching closer to the era of mass consumption and consumerism.
Hence, the growth rate of Germany and Japan was higher than that of the United States.





\subsection*{Question 5:}
\textbf{Which economic variable is used in India to estimate poverty line?
Explain any two limitations of this variable.}

Consumption expenditure is the economic variable used in India to estimate the poverty line where,
\begin{equation}
    \text{Poverty line} = \text{Minimum level of consumption expenditure required to meet the basic needs of the households}
\end{equation}

The two limitations of this variable are:
\begin{itemize}
    \item Informal and non-market transactions, which form a significant portion of economic activities 
          in a country like India, are not incorporated in the consumption expenditure, and cannot be with the present tools of measurement.
          This leads to an significant underestimation of the poverty line.
    \item The differences in the quality of the goods and services consumed by different househlods is also not taken into account.
          The consumption expenditure is calculated by taking the prices of all the goods and services consumed, but clearly,
          such differences dicate the quality of life of the consumers and generalising it will 
          lead to an artificial estimation of the poverty line.
\end{itemize}
Apart from these, the poverty line is also not adjusted for the lack of overall well-being, by
not considering the lack of access to basic amenities like education, health, etc. or accounting for the inequalities within a family 
(like the female child having lesser access to resources,etc.) and the society as a whole (like the caste system,etc.).


\end{document}