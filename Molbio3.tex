\documentclass{report}
\usepackage{titlesec}
\usepackage{titling}
\usepackage{graphicx}
\usepackage{tikz}
\usepackage{pgfplots}
\usepackage{caption}
\usepackage{multirow}
\usepackage{amsmath}

\title{\textbf{UBL201-L Introductory Biology III - Molecular Biology}}
\author{Rahul Chavan - 22086}
\date{26th October 2023}


\usepackage[left=1in, right=1in, top=0.5in, bottom=0.5in]{geometry}
\renewcommand{\maketitle}{
 \begin{center}
    \includegraphics[width=2cm]{IISc_Master_Seal_Black.jpg}
    \vspace{0.5cm}

    \Large
    \textbf{\thetitle}
    
    \vspace{0.5cm}
    
    \Large
    \theauthor
    
    \vspace{0.2cm}
    
    \large
    \thedate

    \vspace{0.5cm}

    \hrule  
    
  \end{center}
}

\pgfplotsset{compat=1.18}
\begin{document}
\maketitle
\begin{center}
    \Large
    \textbf{Is UvrB essential to repair UV-induced DNA damage in Escherichia coli?}
\end{center} 

%Lab report: 
%   Aim of the experiment
%	Principle
%	Results and Discussion: discuss the results (Photograph of the plate) and speculate on the potential source of error.
%   Answer the following questions:

\section*{Aim of the experiment:}
\begin{itemize}
  \item To check if UvrB is essential for the repair of UV-induced DNA damage in E.coli cells.
  \item To study the various kinds of mutations(damage) caused by different wavelengths of UV light. 
  \item To study the various mechanisms by which DNA repair occurs post UV damage.
\end{itemize}

\section*{Principle:}
The principle of this experiment is to study whether UvrB is indispensable for the dark repair (NER) of DNA
damage induced by UV light.
The UV light causes the formation of thymidine dimers in the DNA which do not fit into the double helix and hinder replication and gene expression. 
One of the ways by which these thymidine dimers are repaired is by the UvrABC system. 
The UvrABC system is a nucleotide excision repair system. The UvrA protein scans the DNA for the presence of thymine dimers. 
The UvrB protein binds to the UvrA protein and the UvrC protein binds to the UvrB protein. The UvrC protein then cleaves 
the DNA at the 8th phosphodiester bond from the 5' end and the 4th phosphodiester bond from the 3' end of the thymidine dimer. 
The UvrD protein (a helicase) then removes the damaged DNA strand and the DNA polymerase I fills the gap. The DNA ligase then seals the nick. 
The UvrB protein is essential for the repair of the thymine dimers. If the UvrB protein is not present, the thymidine dimers 
will not be repaired and the cell will die (if other repair mechanisms are also impaired). 
The UvrB protein is essential for the repair of UV-induced DNA damage in E.coli cells.

The mechanism of photoreactivation is hindered in our experiment by performing it in the dark.




\section*{Results and discussions:}
We performed the experiment by making 10-fold serial dilutions up to 10\textsuperscript{-5} of both E.coli TG1 cells and
E.coli TG1 $\Delta$UvrB cells and plating 2$\mu$l of each dilution on LB agar plates. The plates were then exposed to UV light for 0s, 5s, 10s and 15s and 20s.
The plates were then incubated at 37$^{\circ}$C for 24 hours whilst ensuring darkness throughout the process such that photoreactivation is prevented. The results are shown in the table below:


\begin{center}
    \begin{tabular}{|c|c|c|c|c|c|c|c|c|c|c|}
        \hline
        \multirow{2}{*}{Concentration of E.coli cells} & \multicolumn{5}{c|}{Exposure: wild type} & \multicolumn{5}{c|}{Exposure: knockout} \\
        \cline{2-11}
         & 0s & 5s & 10s & 15s & 20s & 0s & 5s & 10s & 15s & 20s \\
        \hline
        10\textsuperscript{-1} & + & + & + & + & + & + & + & - & - & + \\
        \hline
        10\textsuperscript{-2} & + & + & + & + & + & + & - & - & - & - \\
        \hline
        10\textsuperscript{-3} & + & + & + & + & + & + & - & - & - & - \\
        \hline
        10\textsuperscript{-4} & + & + & + & + & - & + & - & - & - & - \\
        \hline
        10\textsuperscript{-5} & + & + & + & - & - & + & - & - & - & - \\
        \hline
        
    \end{tabular}
\end{center}

The '+' sign indicates that the E.coli cells survived and the '-' sign indicates that the E.coli cells died(or did not grow).
We can consistently see that the E.coli TG1 cells survived for most exposure times and the E.coli TG1 $\Delta$UvrB cells died for most exposure times.
This shows that the UvrB protein is essential for the repair of UV-induced DNA damage in E.coli cells.




\section*{Images:}

\begin{figure}[!ht]
    \centering
    \includegraphics[width=4cm]{molbiolab03_0s.jpg}
    \caption*{Figure 1: E.coli TG1 and TG1 $\Delta$UvrB cells exposed to UV light for 0s}
\end{figure}

\begin{figure}[!ht]
    \centering
    \includegraphics[width=4cm]{molbiolab03_05s.jpg}
    \caption*{Figure 2: E.coli TG1 and TG1 $\Delta$UvrB cells exposed to UV light for 5s}
\end{figure}

\begin{figure}[!ht]
    \centering
    \includegraphics[width=4cm]{molbiolab03_10s.jpg}
    \caption*{Figure 3: E.coli TG1 and TG1 $\Delta$UvrB cells exposed to UV light for 10s}
\end{figure}

\begin{figure}[!ht]
    \centering
    \includegraphics[width=4cm]{molbiolab03_15s.jpg}
    \caption*{Figure 4: E.coli TG1 and TG1 $\Delta$UvrB cells exposed to UV light for 15s}
\end{figure}

\begin{figure}[!ht]
    \centering
    \includegraphics[width=4cm]{molbiolab03_20s.jpg}
    \caption*{Figure 5: E.coli TG1 and TG1 $\Delta$UvrB cells exposed to UV light for 20s}
\end{figure}



\section*{Sources of error:}

There are certain dilutions in wild type cells which due not show growth.
This is found only in very low diluutions, hence it can be attributed to lack of cells in the dilution due to which no growth was observed.

However, in the case of the knockout cells, there are certain dilutions which show growth.
This has occured in cells with high concentrations. This can be attributed to the fact that the cells were not exposed to UV light properly as they 
were shielded by other cells due to high cell number.

In case of 20s exposure, the cells were exposed to UV light for 20s but they were also exposed to the light for a few seconds while the plate was being removed from the hood.
This could have caused the cells to repair the damage caused by the UV light and hence show growth. Again this has occured in cells with high concentrations.





\section*{Answer the following questions:}

\subsubsection*{Question 1:}
\textbf{How does nucleotide excision repair differ from photoreactivation?}

\subsubsection*{Answer:}
Nucleotide excision repair is a mechanism by which DNA damage is repaired by removing the damaged DNA strand and replacing it with a new strand.
The NER mechanism is controlled by the UvrABC system.
First, a UvrA-UvrB complex scans the DNA, with the UvrA subunit recognizing distortions in the helix,
caused for example by pyrimidine dimers. When the complex recognizes such a distortion, 
the UvrA subunit leaves and an UvrC protein comes in and binds to the UvrB monomer and, hence, forms a new UvrBC dimer. 
DNA helicase II (or UvrD) then comes in and removes the excised segment by actively breaking 
the hydrogen bonds between the complementary bases. The resultant gap is then filled in using DNA polymerase I and DNA ligase


Whereas, photoreactivation is a mechanism by which DNA damage is repaired by breaking the thymidine dimers using light of wavelength 380-500nm.
It is controlled by DNA photolyases which recognize the “kink” in the DNA, and bind to the site. When excited by blue light (380-500 nm wavelength), the photolyase along with its co-factor FADH- 
traps the energy from photons and transfers it to pyrimidine dimer and eventually break apart the dimer.

Hence, the main difference between NER and photoreactivation is that NER is a dark repair mechanism whereas photoreactivation is a light repair mechanism
where different enzymes are used to repair the damage caused by UV light.







\subsubsection*{Question 2:}
\textbf{Which DNA polymerase is used in nucleotide excision repair (Both in prokaryotes and mammalian cells)?}

\subsubsection*{Answer:}
In prokaryotes, DNA polymerase I (although $\kappa$ or III can also substitute for it) is used in nucleotide excision repair whereas
in eukaryotes, DNA polymerase $\delta$, $\epsilon$ and/or $\kappa$ are used in nucleotide excision repair.




\end{document}