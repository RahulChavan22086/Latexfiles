\documentclass{report}
\usepackage{titlesec}
\usepackage{titling}
\usepackage{graphicx}
\usepackage{tikz}
\usepackage{pgfplots}
\usepackage{caption}
\usepackage{multirow}

\title{\textbf{UBL201-L Introductory Biology III - Immunology}}
\author{Rahul Chavan - 22086}
\date{12th September 2023}


\usepackage[left=1in, right=1in, top=0.5in, bottom=0.5in]{geometry}
\renewcommand{\maketitle}{
 \begin{center}
    \includegraphics[width=2cm]{IISc_Master_Seal_Black.jpg}
    \vspace{0.5cm}

    \Large
    \textbf{\thetitle}
    
    \vspace{0.5cm}
    
    \Large
    \theauthor
    
    \vspace{0.2cm}
    
    \large
    \thedate

    \vspace{0.5cm}

    \hrule  
    
  \end{center}
}

\pgfplotsset{compat=1.18}
\begin{document}
\maketitle
\begin{center}
    \large
    \textbf{Detection and determination of the concentration of an antibody by an indirect ELISA}
\end{center} 

%Lab report: 

%•      Title of the experiment 

%•      Principle 

%•      Results and discussion 

%•      Interpretation 

%•      Precautions 

\section*{Principle:} 
ELISA (enzyme linked immunosorbent assay) is a laboratory technique used to detect the presence of antigens (Ag) or
antibodies (Ab) in a sample. The ELISA has been used as a diagnostic tool in medicine and plant pathology, as well as a
quality-control check in various industries. In simple terms, in ELISA, an unknown amount of antigen is affixed to a surface,
and then a specific antibody is applied over the surface so that it can bind to the antigen. This antibody is linked to an enzyme,
and in the final step a substance is added that the enzyme can convert to some detectable signal, most commonly a colour
change in a chemical substrate. The substrate is converted to a detectable
signal by the enzyme. The amount of signal is directly proportional to the amount of antibody bound to the antigen. The
antibody concentration can be determined by comparing the signal obtained from the sample to the standard curve. The
standard curve is generated by plotting the known concentrations of the antibody against the signal obtained from each
standard. The signal is usually measured by absorbance at 450 nm. The ELISA is a rapid test used for detecting or
quantifying antibody (Ab) against viruses, bacteria and other materials or antigen (Ag). 
In principle, there are two possible approaches to the implementation of an ELISA test: Direct
ELISA and Indirect ELISA.
\subsubsection*{Direct ELISA:} 
Direct ELISA involves directly detecting antigen in the test samples by antibodies (primary/capture antibodies).
 These antibodies are directed against the antigen being sought and are bound to the microtitre plate. The antigen binds the
  antibodies and can be detected by a secondary antibody-enzyme conjugate.
\subsubsection*{Indirect ELISA:}
Indirect ELISA involves detecting the antibody specific for the antigen, which in turn, is bound to the plate.
 The antibody being detected binds the antigen and can be detected in a second step by a secondary antibody-enzyme conjugate. 
 The capture antibody needs to be different from the antibody used for detection. Often a monoclonal antibody is used to 
capture and a polyclonal antiserum is used for detection.
The indirect ELISA is useful for screening specific anitbodies.
  \begin{figure}[htbp]  
   \centering 
   \includegraphics[width=0.7\textwidth]{indirectelisa.jpg} 
   \caption{Indirect ELISA}
   \label{fig: Indirect ELISA } 
  \end{figure}

  \vspace{2cm}

\subsubsection*{Detection principle using Hydrogen peroxidase (HRP)-TMB/$H_{2}O_{2}$ }
The amount of primary antibody bound will be detected using a secondary antibody conjugated to the 
enzyme, Horse radish peroxidase (HRP). Upon incubation with the substrate i.e., hydrogen peroxide ($H_{2}O_{2}$),
 HRP acts on $H_{2}O_{2}$ to release nascent oxygen, which acts on the chromogen, Tetramethylbenzidine (TMB) resulting 
 in the formation of a blue coloured product. The reaction is stopped using 1M sulphuric acid ($H_{2}SO_{4}$), addition 
 of which leads to formation of yellow color which is read at 450 nm in spectrophotometer (ELISA reader).
 


\section*{Results and discussion}
The following are the readings obatined from the spectrophotometer. The readings are in the form of absorbance at 450 nm.


\begin{figure}[htbp]  
  \centering 
  \includegraphics[width=0.5\textwidth]{spectrophotometerreadings.jpg} 
  \caption{spectrophotometer readings}
  \label{fig: readings } 
 \end{figure}
 
  \begin{figure}[htbp]  
    \centering 
    \includegraphics[width=0.2\textwidth]{beforerxn.jpg} 
    \caption{Sample after adding substrate}
    \label{fig: sample with substrate }
  \end{figure}

  \begin{figure}[htbp]  
    \centering 
    \includegraphics[width=0.2\textwidth]{afterrxn.jpg} 
    \caption{Sample after adding stop solution}
    \label{fig: sample with stop solution }
  \end{figure}
 \vspace{5cm}


 \vspace{0.5cm}
 \begin{figure}
  \centering 
  \includegraphics[width=0.5\textwidth]{antibody concentration.jpg} 
  \caption{antibody concentration plot} 
  \label{fig: plot } 
 \end{figure}

\subsubsection*{Antibody concentration}
The antibody concentration is calculated using the following formula:
 \begin{equation}
   \text{Concentration of antibody(in mg/mL)} = \frac{\text{Concentration of antibody from graph (ng/mL)}}{10^{6}} \times {\text{Dilution Factor}}
 \end{equation}



From the plot we get the concentration of the test samples as follows:

Test sample 1 = \textbf{0.0582 mg/mL}

Test sample 2 = \textbf{0.0256 mg/mL}

Test sample 3 = \textbf{0.0116 mg/mL}

Hence, the concentration of the antibody in the test samples is 0.0582 mg/mL, 0.0256 mg/mL and 0.0116 mg/mL respectively. We could see that the absorbance from the blank was extremely lowerconfirming that our negative
control worked well. The absorbance values of the test samples were higher than the blank confirming that the test samples were also working well. 
The plot was so used as there was linear relationship between the concentration of the antibody and the absorbance values.
Using the equation of the line y = mx + b, we can determine the concentration of the unknown sample.

\section*{Interpretation:}

We can see a decrease in the concentartion of the antibody as we go from test sample 1 to test sample 3. 
This suggests that the test sample 1 is at a higher risk of being exposed to the antigen than the other two test samples.
This could be due to the fact that the test sample 1 was exposed to the antigen for a longer duration than the other two test samples.

\section*{Precautions:}

\begin{itemize}
  \item Wear appropriate personal protective equipments, including lab coats, gloves, and safety goggles,
         to protect against chemical exposure.
  \item Use sterile techniques and clean glassware to prevent contamination of samples and reagents.
  \item Calibrate and standardize the spectrophotometer using appropriate standards and 
         controls to ensure accurate measurements.
  \item Avoid pippeting errors and ensure that the pippetes are calibrated.
  \item Keep everything properly thawed as the samples and reactions are temperature sensitive.
  \item Perform duplicates and triplicates to reduce the error and enhance reproducibility of results.
  \item Careful disposal of bioharzardous waste.
\end{itemize}


\end{document}